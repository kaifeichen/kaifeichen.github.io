\documentclass[margin,line]{res}

\usepackage{hyperref}
% The following metadata will show up in the PDF properties
\hypersetup{
  colorlinks = true,
  urlcolor = black,
  pdfauthor = {Kaifei Chen},
  pdfkeywords = {},
  pdftitle = {Kaifei Chen},
  pdfsubject = {Curriculum Vitae},
  pdfpagemode = UseNone
}

\oddsidemargin -.5in
\evensidemargin -.5in
\textwidth=6.0in
\itemsep=0in
\parsep=0in

\newenvironment{list1}{
  \begin{list}{\ding{113}}{
      \setlength{\itemsep}{0in}
      \setlength{\parsep}{0in} \setlength{\parskip}{0in}
      \setlength{\topsep}{0in} \setlength{\partopsep}{0in}
      \setlength{\leftmargin}{0.17in}}}{\end{list}}
\newenvironment{list2}{
  \begin{list}{$\bullet$}{
      \setlength{\itemsep}{0in}
      \setlength{\parsep}{0in} \setlength{\parskip}{0in}
      \setlength{\topsep}{0in} \setlength{\partopsep}{0in}
      \setlength{\leftmargin}{0.2in}}}{\end{list}}


\begin{document}

  \name{Kaifei Chen \vspace*{.1in}}

  \begin{resume}

    \section{\sc Contact Information}

      \vspace{.05in}
      \begin{minipage}{0.45\linewidth}
        RADLab, Room 465 Soda Hall \#1776 \\
        University of California, Berkeley \\
        Berkeley, CA 94720-1776
      \end{minipage}
      \begin{minipage}{0.45\linewidth}
        \begin{tabular}{ll}
          \textit{Voice}: & +1 (510)883-3298 \\
          \textit{E-mail}: & \href{mailto:kaifei@berkeley.edu}{\tt kaifei@berkeley.edu} \\
          \textit{WWW}: & \url{http://www.cs.berkeley.edu/~kaifei/}
        \end{tabular}
      \end{minipage}

    \section{\sc Current Status}

      {\bf Ph.D. Student} \\
      Computer Science  \\
      University of California, Berkeley

    \section{\sc Research Interests}

      Wireless and Mobile Systems, Sensor Networks, Ubiquitous Computing

    \section{\sc Education}
%
      %{\bf Microsoft Research Asia}, Beijing, China \\
      %\vspace*{-.1in}
      %\begin{list1}
      %  \item[] B.S. Computer Science \hfill Sept. 2007 - Present \\
      %      (first three years) \\
      %      GPA: 3.68 \\
      %      GPA Rank: 5/121
      %\end{list1}

      {\bf University of California, Berkeley}, Berkeley, CA \\
      \vspace*{-.1in}
      \begin{list1}
        \item[] Ph.D. Computer Science \hfill Aug. 2012 - Present \\
      \end{list1}

      \vspace*{-5.5mm}
      {\bf University of Science and Technology of China}, Hefei, Anhui China \\
      \vspace*{-.1in}
      \begin{list1}
        \item[] B.Eng. Computer Science and Technology \hfill Sept. 2007 - Jul. 2012 \\
     %       (first three years) \\
     %       GPA: 3.68 \\
     %       GPA Rank: 5/121
      \end{list1}

    \section{\sc Honors and Awards}

      Excellent Intern Award at Microsoft Research Asia \hfill 2011\\
      {\it in top 30\% among all Microsoft Research Asia interns}

      \vspace*{-2.5mm}
      ACM SenSys 2011 Best Demo Award \hfill 2011\\
      {\it in best 2 among 33 demos}

      \vspace*{-2.5mm}
      ACM UbiComp 2011 Best Demo Award \hfill 2011\\
      {\it best among 36 demos}

      \vspace*{-2.5mm}
      2nd Prize of China Aerospace Science and Technology Corporation (CASC) Scholarship\hfill 2010\\
      {\it in top 3\% among about 160 students}

      \vspace*{-2.5mm}
      2nd Prize of Contemporary Undergraduate Mathematical Contest in Modeling (CUMCM) \hfill 2010\\
      {\it in top 7.9\% among 14108 teams}

      \vspace*{-2.5mm}
      Outstanding Student Scholarship Grade 1 at University of Science and Technology of China \hfill 2009\\
      {\it in top 5\% among about 160 students}

      \vspace*{-2.5mm}
      Outstanding Student Scholarship Grade 3 at University of Science and Technology of China \hfill 2008\\
      {\it in top 20\% among about 510 students}

    \section{\sc Publications}
      Ben Zhang, \underline{Kaifei Chen}, Yun Cheng, Chieh-Jan Mike Liang, Xiaofan Jiang, and Feng Zhao. ``{\it Location-log: Bringing Online Shopping Benefits to the Physical World with Magnetic-based Proximity Detection}". In Proceedings of the 2nd International Workshop on Mobile Sensing, Apr. 2012. Beijing, China.

      \vspace*{-2.0mm}
      Chieh-Jan Mike Liang, \underline{Kaifei Chen}, Jie Liu, Nissanka Bodhi Priyanthaz, and Feng Zhao. ``{\it Poster Abstract: Shipping Data from Heterogeneous Protocols on Packet Train}". In Proceedings of the 11th ACM/IEEE Conference on Information Processing in Sensor Networks (IPSN 2012), pages 127-128, Apr. 2012. Beijing, China.

      \vspace*{-2.0mm}
      Xiaofan Jiang, Chieh-Jan Mike Liang, \underline{Kaifei Chen}, Ben Zhang, Jeff Hsu, Bin Cao, Jie Liu, and Feng Zhao. ``{\it Design and Evaluation of a Wireless Magnetic-based Proximity Detection Platform for Indoor Applications}". In Proceedings of the 11th ACM/IEEE Conference on Information Processing in Sensor Networks (IPSN 2012), pages 221-232, Apr. 2012. Beijing, China. (Acceptance Rate: 15.0\%)

      \vspace*{-2.0mm}
      Xiaofan Jiang, Chieh-Jan Mike Liang, Feng Zhao, \underline{Kaifei Chen}, Jeff Hsu, Ben Zhang, and Jie Liu. ``{\it Demo: Creating Interactive Virtual Zones in Physical Space with Magnetic-Induction}". In Proceedings of the 9th ACM Conference on Embedded Networked Sensor Systems (SenSys 2011), pages 431-432, Nov. 2011. Seattle, WA. {\bf (Best Demo Award)}

      \vspace*{-2.0mm}
      Zheng Sun, Aveek Purohit, \underline{Kaifei Chen}, Shijia Pan, Trevor Perring, and Pei Zhang. ``{\it PANDAA: Physical Arrangement Detection of Networked Devices through Ambient-Sound Awareness}". In Proceedings of the 13th ACM International Conference on Ubiquitous Computing (UbiComp 2011), pages 425-434, Sept. 2011. Beijing, China. (Acceptance Rate: 16.6\%) {\bf (Affiliated demo wins the Best Demo Award)}

      %\vspace*{-2.0mm}
      %\underline{Kaifei Chen}, Dongqu Chen, Daiwei Zhang and Yufei Liu. ``{it Analysis on Wide Use of Electric Vehicles}". The Mathematical Contest in Modeling/ The Interdisciplinary Contest in Modeling (MCM/ ICM 2011), Feb. 2011.
%
      %\vspace*{-2.0mm}
      %\underline{Kaifei Chen}, Yufei Liu, and Daiwei Zhang. ``{\it Modification and Calibration of Capacity Tables of Tilted Underground Oil Tanks}". (in Chinese) The 19th Contemporary Undergraduate Mathematical Contest in Modeling (CUMCM 2010), Nov. 2010.
%
    %\section{\sc Papers under submission}
%
    %One full paper has been submitted to the 10th International Conference on Mobile Systems, Applications and Services (MobiSys 2012). {\bf (Available on request)}
    %Aveek Purohit, Zheng Sun, Shijia Pan, \underline{Kaifei Chen}, and Pei Zhang. ``SugarTrail : Mapless Navigation in Dynamic Indoor Environments with Resource-limited Mobile Devices". The 10th ACM Conference on Embedded Networked Sensor Systems (SenSys 2012), Jun. 2012.
%
    %\section{\sc Papers in preparation}
%
    %One full paper is being prepared for the 10th ACM Conference on Embedded Networked Sensor Systems (SenSys 2012).
%
    \section{\sc Academic Experience}

      {\bf University of California, Berkeley}, Berkeley, CA

      \vspace{-.3cm}
      {\em Graduate Student Researcher} \hfill Aug. 2012 - Present\\
      Supervisor: Prof. David E. Culler and Prof. Randy H. Katz\\

      \vspace*{-.1in}
      \begin{list1}
        \item[] {\bf Software Defined Buildings}\\
%        This is a new project I participate in. It is a data center asset tracking system using RFID and wireless sensors. I am involved in developing the software for the top-of-rack radio antenna module (RAM) to collect data and events from each racks to central servers.
%        \vspace*{.05in}
%        \begin{list2}
%          \item Developed the network protocols that efficiently transfer large amounts of data from RFID readers to back-end servers over low-power ZigBee backbones.
%        \end{list2}
      \end{list1}

      {\bf Wicresoft Co., Ltd.}, Beijing, China

      \vspace{-.3cm}
      {\em Software Developer Intern} \hfill Oct. 2011 - Jun. 2012\\
      Supervisor: Dr. Chieh-Jan Mike Liang and Dr. Xiaofan Jiang\\

      \vspace*{-.1in}
      \begin{list1}
        \item[] {\bf Large-scale RFID-based Asset-tracking System}\\
        This is a new project I participate in. It is a data center asset tracking system using RFID and wireless sensors. I am involved in developing the software for the top-of-rack radio antenna module (RAM) to collect data and events from each racks to central servers.
        \vspace*{.05in}
        \begin{list2}
          \item Developed the network protocols that efficiently transfer large amounts of data from RFID readers to back-end servers over low-power ZigBee backbones.
          \item Developed driver of ZigBee radio AT86RF231 in C
        \end{list2}
      \end{list1}

      {\bf Microsoft Research Asia}, Beijing, China

      \vspace{-.3cm}
      {\em Full-time Research Intern} in Mobile and Sensing Systems group \hfill Apr. 2011 - Sept. 2011\\
      Mentor: Dr. Chieh-Jan Mike Liang\\

      \vspace*{-.1in}
      \begin{list1}
        \item[] {\bf LiveSynergy}\\
        LiveSynergy is a novel magnetic-based proximity sensing and human-environment interacting system. It consists of three parts: LivePulses (devices to give virtual zones using magnetic induction), LiveLinks (devices to discover and interact with LivePulses), interoperability platform (wireless links and IPv6-based interfaces that allow rich human-environment interactions).
        \vspace*{.05in}
        \begin{list2}
          \item Developed a magnetic data transmission system between LivePulses and LiveLinks in TinyOS, from hardware driver to high-level API and applications.
          \item Developed the  interoperability platform in TinyOS on both LivePulses and LiveLinks and in Python on server, using 802.15.4c radios, single-hop network topologies, and IPv6.
          \item Improved the interoperability platform with regard to robustness and efficiency, involving aspects of duty cycle, packets caching, and CSMA.
          \item Conducted several experiments to measure the performance of LiveSynergy and a real world deployment.
          \item Presented several demos for LiveSynergy, including MSRA Open House for UbiComp 2011 and MSRA 2011 Technical Advisory Board Review.
        \end{list2}
      \end{list1}

      {\bf Carnegie Mellon University}, {\bf Silicon Valley Campus}, Mountain View, CA

      \vspace{-.3cm}
      {\em Visiting Student} in Department of Electrical and Computer Engineering \hfill Sept. 2010 - Mar. 2011\\
      Advisor: Prof. Pei Zhang\\
      \vspace*{-.1in}
      \begin{list1}
        \item[] {\bf PANDAA}\\
        PANDAA (Physical Arrangement Detection of Networked Devices through Ambient-Sound Awareness) is a novel low-cost acoustic indoor localization system. It needs no pre-deployed infrastructure, is readily applicable for off-the-shelf  mobile devices, processes fully automatic detection and achieves high accuracy.
        \vspace*{.05in}
        \begin{list2}
          \item Designed and developed an algorithm to iteratively estimate distances between all pairs in a microphone network purely using Time Difference of Arrival from ambient sound sources to microphones.
          \item Designed and developed a greedy algorithm to compute the relative microphone locations, making the most use of the information by giving more priorities to more precise pair distance estimations.
          %\item Implemented some experiments in a meeting room to verify and optimize the algorithms.
          \item Designed and implemented a metric to evaluate the system state when it is running.
          \item Presented a demo for PANDAA in UbiComp 2011.
        \end{list2}


        \vspace*{.1in}
        \item[] {\bf SensorFly}\\
        SensorFly is a novel low-cost controlled-mobile aerial sensor networking platform. It is light and small helicopter robot integrated with flight control, information acquisition and communication systems. A swarm of them can set up a wireless distributed network dynamically and execute tasks in cooperation.% It can be applied to searching survivors in emergencies like fire or earthquake or detecting unknown environments in exploration or combat.
        \vspace*{.05in}
        \begin{list2}
          \item Developed the communication system of SensorFly, including packet sending/receiving and round-trip time calculation of radio signal transmission.
          \item Developed the navigation system of SensorFly, which enables a swarm of the helicopters to explore unknown environments and navigate in collaboration by constructing a virtual map consisting of several landmark clusters.% A signature is a combination of the Round-Trip Time of signal transmission between a helicopter and several beacons, which is constant for our radio.
          \begin{list2}
            \item[-] Designed, verified and optimized the navigation algorithm.
            \item[-] Deployed the navigation system in a supermarket, collected data at every grid point and simulated navigation based on these data in Matlab.
            \item[-] Designed and developed a Maemo 5 (Nokia N900) application for the navigation system in C++, using QT, sockets and multiple threads.
          \end{list2}
        \end{list2}

      \end{list1}
%
%      {\bf University of Science and Technology of China}, Hefei, Anhui China
%
%      \vspace{-.3cm}
%      {\em Undergraduate} in School of Computer Science and Technology \hfill Sept. 2007 - Present\\
%      Selected course projects are:
%      \vspace*{.05in}
%      \begin{list2}
%        \item MIPS R4000PC CPU simulation in Verilog, including eight-stage pipeline, TLB, cache, memory and some interruption/exception handling parts.
%        \item Back end of a high-level language compiler in Java, transforming abstract syntax tree (Eclipse AST) to intermediate language (three address code), then to assembly language (GNU Assembler).
%        \item Classifier in C++, using decision tree, post-pruning and theory of receiver operating characteristic (ROC) curve.
%        \item Basic computer architecture simulation in C++, including cache, memory, I/O, ALU and instruction execution.
%        \item Chatroom client and server for Linux in C++, using sockets, POSIX threads and gtkmm.
%        \item Bank queue scheduling system for Linux in C++, using sockets and multiple processes.
%        \item Washing machine controller in VHDL, which can create signals controlling motor and monitor.
%      \end{list2}

    \section{\sc Skills}
      C, C++, Java, C\#, Matlab, Python, nesC, VHDL, Verilog, assembler (X86, MIPS), SQL, PHP, HTML, QT, Latex, Eclipse, Linux, TinyOS.

    \section{\sc References}
      Available on request.

  \end{resume}

\end{document}
%
%
%\documentclass[margin,line]{res}
%
%
%\usepackage{hyperref}
% The following metadata will show up in the PDF properties
%\hypersetup{
%  colorlinks = true,
%  urlcolor = black,
%  pdfauthor = {Kaifei Chen},
%  pdfkeywords = {},
%  pdftitle = {Kaifei Chen},
%  pdfsubject = {Curriculum Vitae},
%  pdfpagemode = UseNone
%}
%
%\oddsidemargin -.5in
%\evensidemargin -.5in
%\textwidth=6.0in
%\itemsep=0in
%\parsep=0in
%
%\newenvironment{list1}{
%  \begin{list}{\ding{113}}{%
%      \setlength{\itemsep}{0in}
%      \setlength{\parsep}{0in} \setlength{\parskip}{0in}
%      \setlength{\topsep}{0in} \setlength{\partopsep}{0in}
%      \setlength{\leftmargin}{0.17in}}}{\end{list}}
%\newenvironment{list2}{
%  \begin{list}{$\bullet$}{%
%      \setlength{\itemsep}{0in}
%      \setlength{\parsep}{0in} \setlength{\parskip}{0in}
%      \setlength{\topsep}{0in} \setlength{\partopsep}{0in}
%      \setlength{\leftmargin}{0.2in}}}{\end{list}}
%
%
%\begin{document}
%
%\name{Christopher J. Paciorek \vspace*{.1in}}
%
%\begin{resume}
%\section{\sc Contact Information}
%\vspace{.05in}
%\begin{tabular}{@{}p{2in}p{4in}}
%Baker Hall 232             & {\it Voice:}  (412) 268-6276 \\
%Department of Statistics   & {\it Fax:}    (412) 268-7828 \\
%Carnegie Mellon University & {\it E-mail:}  paciorek@stat.cmu.edu\\
%Pittsburgh, PA  15213 USA  & {\it WWW:} www.stat.cmu.edu/\verb+~+paciorek \\
%\end{tabular}
%
%
%\section{\sc Research Interests}
%Bayesian statistics, spatial statistics, nonparametric regression,
%statistical methods for large datasets, statistics for public policy
%
%\section{\sc Education}
%{\bf Carnegie Mellon University}, Pittsburgh, Pennsylvania USA\\
%%{\em Department of Statistics}
%\vspace*{-.1in}
%\begin{list1}
%\item[] Ph.D. Candidate, Statistics, December 2001 (expected
%  graduation date: May 2003)
%\begin{list2}
%\vspace*{.05in}
%\item Dissertation Topic:  ``Nonstationary Covariance Models for
%  Spatial Data and Regression Problems''
%%\item Dissertation Topic:  ``Hierarchical Models for Multiple Ratings
%%  in Performance-Based\\ \hspace*{1.23in} Student Assessments.''
%\item Advisor:  Mark J. Schervish
%\end{list2}
%\vspace*{.05in}
%\item[] M.S., Statistics,  May 2000
%\end{list1}
%
%{\bf Duke University}, Durham, North Carolina USA\\
%%{\em Department of Mathematics and Statistics}
%\vspace*{-.1in}
%\begin{list1}
%\item[] M.S., Botany (Ecology),  May, 1998
%\end{list1}
%
%{\bf Carleton College}, Northfield, Minnesota USA\\
%%{\em Department of Mathematics and Statistics}
%\vspace*{-.1in}
%\begin{list1}
%\item[] B.A., Biology,  May, 1993
%\end{list1}
%
%
%\section{\sc Honors and Awards}
%National Science Foundation Graduate Research Fellowship, 1996
%
%%\vspace*{-2.5mm}
%%NSF Vertical Integration of Research and Education in Statistics and
%%Mathematical Sciences\\ (VIGRE) teaching fellowship.
%%
%\vspace*{-2.5mm}
%%%Carleton College: graduated Magna Cum Laude, Honors in Biology, Phi Beta Kappa, 1993
%
%\section{\sc Academic Experience}
%{\bf Carnegie Mellon University}, Pittsburgh, Pennsylvania USA
%
%\vspace{-.3cm}
%{\em Graduate Student} \hfill {\bf August, 1998 - present}\\
%Includes current Ph.D.~research, Ph.D.~and Masters level coursework and
%research/consulting projects.
%
%%\vspace{-.1cm}
%{\em Instructor} \hfill {\bf May - June, 2002}\\
%Co-taught graduate level course for the Master of Science in
%Computational Finance program.  Shared responsibility for lectures, exams,
%homework assignments, and  grades.
%\vspace*{.05in}
%\begin{list2}
%\item 46-731 Probability and Statistics, Summer 2002.
%\end{list2}
%
%
%%\vspace{-.1cm}
%{\em NSF VIGRE Teaching Fellow} \hfill {\bf January - May, 2001}\\
%Head teaching assistant.
%Duties included  shared administrative responsibilities with faculty
%instructor, fielding of all student inquiries, and oversight of
%graduate student teaching assistants and graders.
%\vspace*{.05in}
%\begin{list2}
%\item 36-217 Probability Theory and Random Processes, Spring 2001.
%\end{list2}
%
%%\vspace{-.1cm}
%{\em Teaching Assistant} \hfill {\bf August, 2001  - present}\\
%Duties at various times have included
%office hours and leading weekly computer lab exercises.
%
%
%
%\section{\sc Publications}
%Paciorek, C.J., J.S. Risbey, V. Ventura, and R.D.Rosen. 2002. Multiple indices of Northern Hemisphere Cyclone
%Activity, Winters 1949-1999. Journal of Climate 15:1573-1590.
%
%Paciorek, C.J., R. Condit, S.P. Hubbell, and R.B. Foster.  2000.
%The demographics of resprouting in tree and shrub species of a moist
%tropical forest.  Journal of Ecology 88:765-777.
%
%Paciorek, C.J., B.R. Moyer, R.A. Levin, and S.L. Halpern.  1995.
%Pollen consumption by hummingbird flower mite {\it Proctolaelaps
%  kirmsei} and possible fitness effects on {\it Hamelia patens}.
%Biotropica 27:258-262.  (author order determined by lot)
%
%\section{\sc Papers in preparation}
%
%Ventura, V., C.J. Paciorek, and J.S. Risbey.  Controlling the proportion of falsely-rejected hypotheses when %conducting multiple tests with geophysical data.
%
%Ickes, K., C.J. Paciorek, and S. Thomas.  Effects of wild pigs on
%forest demographic processes in Malaysia.
%
%\section{\sc Conference Presentations}
%Paciorek, C.J., J.S. Risbey, V. Ventura, and R.D.Rosen.  2001.  Changes in Northern Hemisphere winter storm %activity (1949-1999) based
%on a comparison of cyclone indices.  8th International Meeting on
%Statistical Climatology, Luneberg, Germany, March, 2001.
%
%Paciorek, C.J. and R. Rosenfeld.  2000.  Minimum classification error
%training in exponential language models.  2000 Spring Transcription
%Workshop, College Park, Maryland.
%\vspace*{-.25in}
%\begin{verbatim}http://www.nist.gov/speech/publications/tw00/html/abstract.htm#cp1-50\end{verbatim}
%
%\section{\sc Professional Experience}
%{\bf Bureau of Transportation Statistics, U.S. Department of
%  Transportation}, Washington, District of Columbia USA
%
%\vspace{-.3cm}
%{\em Summer researcher} \hfill {\bf May, 2000 - August, 2000}\\
%Carried out several consulting projects, including modelling of
%injuries to cadavers in crash test experiments, analysis of airline
%delay data, and advice on analysis of airline economics data.
%
%{\bf Abt Associates}, Bethesda, Maryland USA
%
%\vspace{-.3cm}
%{\em Associate Programmer Analyst and Research Assistant} \hfill {\bf
%  October, 1994 - August, 1996}\\
%Researcher and computer model developer for U.S. EPA Regulatory Impact
%Analysis of Section 403 Lead Paint Hazard Rule.  Other projects
%included database analysis, literature reviews, and cost-benefit analysis.
%
%\section{\sc Computer Skills}
%\begin{list2}
%\item Statistical Packages:  R, S-Plus, BUGS; some experience
%  with SAS; extensive use of C and Fortran statistical libraries.
%\item Languages:  C++, Perl, Pascal, some use of Unix shell scripts,
%  MPI parallel processing library.
%\item Applications: Generic Mapping Tools (GMT) - Unix mapping software, \LaTeX, common Windows
%  database, spreadsheet, and presentation software
%\item Algorithms: Experience programming Markov Chain Monte Carlo
%  simulations of Bayesian posterior distributions
%\item Operating Systems:  Unix/Linux, Windows.\\
%\end{list2}
%
%
%
%\end{resume}
%\end{document}
