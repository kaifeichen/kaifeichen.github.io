\documentclass[a4paper, 12pt]{article}

\usepackage[top=0.1in,bottom=1.1in,left=1in,right=1in]{geometry}
\usepackage{fancyhdr}
\usepackage{lastpage}

\makeatletter% since there's an at-sign (@) in the command name

\pagestyle{fancy}
%\fancyhead{}
%\lhead{}
%\chead{}
%\rhead{}
\fancyfoot{}
\lfoot{}
\cfoot{\thepage/\pageref{LastPage}}
\rfoot{}

\author{Kaifei Chen, University of Science and Technology of China}
\title{\textbf{Research Statement}}
\date{}

\begin{document}

\maketitle
%\thispagestyle{fancy}

My research interest is on sensing or distributed systems and networks, like cyber-physical systems, wireless and mobile systems, wireless sensor networks, and ubiquitous computing. With the increasing popularity and infiltration of heterogeneous sensors and mobile devices in human lives, both the potential and challenge of building applications from these distributed devices are pretty huge. We are presented numerous problems from system developing like energy consumption monitoring/saving system and real-time operating system, to strategy issues such as indoor localization algorithms, cloud computing, and large scale network protocols.


I am greatly interested in developing sensing systems combined with novel ideas and techniques, which solve real-world problems and potentially have great impact on human lives. I have worked on a navigation system for a mobile aerial networking platform called SensorFly. This navigation system enables a swarm of the tiny helicopters to explore unknown environments and navigate in collaboration by constructing a virtual map consisting of signature clusters. Moreover, I developed a magnetic-based proximity sensing and human-environment interacting system called LiveSynergy \cite{LiveSynergyDemo}. I am in charge of the network part and hardware drivers, involving magnetic data transmission, 802.15.4c radios, single-hop network topologies, and IPv6 between nodes and servers.


I am also very interested in devising and optimizing algorithms, protocols, and architectures in wireless or distributed networks. I have designed and developed an acoustic indoor localization algorithm called PANDAA \cite{PANDAA} with two team workers. This algorithm estimates distances between microphones only using Time Difference of Arrival (TDoA) from sound sources to microphones. It needs no pre-deployed infrastructure, is readily applicable for off-the-shelf mobile devices, processes fully automatic detection and achieves high accuracy. In addition, I just participated in a new large-scale RFID-based asset-tracking System. I am working on the network protocols that efficiently transfer large amounts of data from RFID readers to back-end servers over low-power ZigBee backbones.

\vfill
\begin{thebibliography}{9}

\bibitem{LiveSynergyDemo} Xiaofan Jiang, Chieh-Jan Mike Liang, \underline{Kaifei Chen}, Ben Zhang, Jeff Hsu, Jie Liu, and Feng Zhao. ``\emph{Demo: Creating Interactive Virtual Zones in Physical Space with Magnetic-Induction}". In the Proceedings of the 9th ACM Conference on Embedded Networked Sensor Systems (SenSys 2011), pages 431-432, Nov. 2011. Seattle, WA.
\bibitem{PANDAA} Zheng Sun, Aveek Purohit, \underline{Kaifei Chen}, Shijia Pan, Trevor Perring, and Pei Zhang. ``\emph{PANDAA: Physical Arrangement Detection of Networked Devices through Ambient-Sound Awareness}". In the Proceedings of the 13th ACM International Conference on Ubiquitous Computing (UbiComp 2011), pages 425-434, Sept. 2011. Beijing, China.
\end{thebibliography}

\end{document}
